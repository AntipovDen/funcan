\documentclass[russian]{article}
\usepackage[T2A,T1]{fontenc}
\usepackage[utf8]{inputenc}
\usepackage[a4paper]{geometry}
\geometry{verbose,tmargin=4cm,bmargin=4cm,lmargin=4cm,rmargin=4cm}
\usepackage{babel}

\title{Про функциональный анализ}
\author{Денис Антипов}

\begin{document}
\maketitle

Если Вам учиться легко, то Вы либо очень способны, либо Вас ничему не учат.
Татьяна Толстая

Анатолий Абрамович Шалыто недавно попросил меня написать текст о том, зачем на кафедре <<Компьютерные технологии>> нужен такой предмет, как функциональный анализ, так как этот этот вопрос возникает среди студентов и даже преподавателей. Существует и противоположная точка зрения: один наш выдающийся выпускник сказал, что никогда не стеснялся того, что закончил ИТМО, так как в программу входили такие предметы как функан и теория функций комплексной переменной.

Меня в первую очередь удивил сам факт того, что у студентов и преподавателей возникает вопрос необходимости функана в учебной программе. Во-первых, для меня вопрос <<зачем нужен функан на КТ>> стоит в одном ряду с вопросом <<зачем нужна математика в школе>>. Он кажется нелепым по той причине, что функциональный анализ является одним из базовых предметов, что пояснено ниже. %Это, вполне вероятно, является субъективным мнением, с которым, тем не менее, согласятся многие выпускники кафедры, но я приведу рассуждения по этому поводу позже.

Вопрос студентов удивил меня также и потому, что он задается несмотря на то, что в наше время у всех есть доступ к интернету, в том числе к ресурсу Google.com. С его помощью легко найти обсуждения данной темы. Например, по запросу <<Why learn functional analysis>> сразу можно выйти на рассуждение~\cite{why-funcan}, в котором приведены примеры применения и причины необходимости функционального анализа в учебной программе по прикладной математике. Данное обсуждение является далеко не единственным в сети на эту тему. Если же вопрос не только про функциональный анализ, а зачем вообще нужна чистая математика в учебных программах, то, опять, по запросу <<why we learn pure math>> можно найти обсуждение и этого вопроса, например,~\cite{why-pure-math}. Правда, для всего этого надо знать английский язык:) А тот, кто его не знает или не хочет знать, может затеять дискуссию, зачем в институте учить иностранный язык, и такое можно устроить с любым предметом, который дается весьма непросто, например, с физвоспитанием.

Далее, меня удивило то, что студенты в течение многих лет не задавли этот вопрос на нашей кафедре, и только новые поколения студентов не стесняются его задавать, причем в такой, например, форме: <<Если я собираюсь делать сайты, то зачем мне нужны функан и диффуры?>> Возникает вопрос, а туда ли он поступил, и мне кажется, что с такой мотивацией он скоро куда-то исчезнет.

И наконец, меня удивило то, что люди не задали этот вопрос много лет преподающему на нашей кафедре функан Николаю Юрьевичу Додонову, который преподает этот предмет не только у нас, но и на МатМехе СПбГУ, или хотя бы мне, так как многие знают, что я имею отношение к преподаванию математики на кафедре. Мы могли бы объяснить или хотя бы посоветовать посмотреть книгу~\cite{kollats}, которая хоть и была издана давно, но приведенные в ней примеры актуальны до сих пор.

Несмотря на это, я все-таки понимаю необходимость написания данного текста. При этом своей задачей я ставлю именно сбор разобщенной информации в одном месте. Хорошо, что меня под рукой есть много источников информации, и я могу просто цитировать людей, которые умнее меня, вместо того, чтобы формулировать какие-то мысли самому. Хотя ближе к концу я добавлю и некоторые собственные рассуждения.

На самом деле вопрос необходимости функана куда более глубокий, чем кажется, и уходит корнями в философские вопросы о методах познания. Для начала стоит вспомнить, что в Новое время (в XVII веке) два основных направления в философии науки --- эмпиризм и рационализм, которые во многом противопоставляли себя друг другу~\cite{phil}. В основе классического рационализма, главные принципы которого были сформулированы Декартом~\cite{descartes}, лежит идея возможности логического познания мира, которая берет свое начало еще из <<Аналитик>> Аристотеля~\cite{aristotel}. Эмпиризм, основателем которого принято считать Бэкона~\cite{bacon}, напротив, считает возможным только чувственное восприятие мира и ставит единственным критерием истинности эксперимент. В наше время большинство ученых сходится в том, что ни экспериментальная, ни теоретическая наука не самодостаточны, а дополняют друг друга, и потому обе являются необходимыми для расширения человеческих знаний. Более подробное рассуждение на эту тему в области эволюционных вычислений можно найти в первой части~\cite{doerr}. Разумеется, стоит признать, что до сих пор даже среди ученых встречаются люди, не признающие чисто теоретические или чисто практические работы (первые встречаются чаще). В этом я убедился на недавней конференции PPSN 2018, где потратил минут десять своего доклада на то, чтобы объяснить одному китайцу необходимость теории в области эволюционных вычислений. Однако десяти минут было недостаточно, чтобы изменить мнение убежденного эмпирика, так как, повторюсь, данные вопросы являются философскими.

Функциональный анализ является, как это ясно из названия, теоретической наукой, как и многие другие ветви математики, такие как топология, теория чисел, теория игр и другие. Однако, несмотря на то, что все эти науки являются неприкладными по своей сути, они все нашли себе применение в прикладных задачах. Топология применяется в анализе данных (TDA --- Topological Data Analysis)~\cite{topology}, теория чисел --- в криптографии~\cite{numbers}, теория игр --- в экономике~\cite{games}. Функциональному анализу также было найдено практическое применение. Самым ярким примером является его применение в квантовой механике~\cite{quantum}. Но большинство современных студентов нашей кафедры не считают нужным изучать квантовую механику (и, как это ни грустно, физику в целом), поэтому более близкий пример для КТ-шников --- применение функана для оценки погрешности вычислений численных методов при решении различных задач, в том числе нелинейных, что описано в книге~\cite{kollats}. В этой же книге содержится много ссылок на другие работы, посвященные практическому применению функционального анализа. В случае, если примеров применения фукана все еще недостаточно, он широко используется в теории вероятностей для анализа стохастических процессов~\cite{prob}. В своей деятельности я пользуюсь функаном именно в этом контексте. Например, в моей последней работе с Бенжамином Доерром~\cite{plateau} знания функционального анализа очень помогали осознавать особенности анализируемого стохастического процесса и получить новые научные результаты, в частности разработать новый метод анализа эволюционных алгоритмов.

Другая причина, почему стоит изучать функан, --- историческая. Стоит признать, что большинство известных математических результатов было получено просто потому, что математикам это было интересно, а не потому, что они знали про какое-любо практическое применение своих результатов заранее, но которое было найдено позже (иногда сразу же, а иногда и через несколько десятилетий). <<Ищите и обрящете>>.

Здесь можно снова привести примеры топологии, теории чисел и теории игр, так как сначала появились эти ветви математики, а только потом уже люди нашли им практическое применение. Но наиболее интересным мне кажется пример Джорджа Буля. Он одним из первых пришел к идее, что математик должен оперировать символами, представляющими некоторые объекты, а не самими объектами. Буль утверждал, что математика не должна привязываться к чему-то реальному и  должна быть абстрактной. Это привело его к разработке матлогики и булевой алгебры в 1847 году~\cite{boole}. И хотя Буль очень хотел, чтобы его алгебра была примером чистой, неприкладной математики, все мы знаем, что после развития вычислительной техники работы Буля стали настолько прикладными, что современный мир просто не мог бы без них существовать. Продолжатель дела Буля --- Клод Элвуд Шеннон закончил MIT по специальности <<электротехника и математика>>, что позволило ему приложить теорию Буля к контактным схемам. Однако я знал выдающихся математиков, которые долго расспрашивали, существенно ли, что диод проводит только в одну сторону. Таким образом, одной математики тоже может быть недостаточно, и именно поэтому в направлении подготовки или специальности нашей кафедры и есть слово <<Прикладная>>. Но прикладная \emph{математика} и информатика. А тот, кто этого не понимает...

С функциональным анализом история была примерно такой же. Хоть он и начинал свое развитие примерно в одно время с квантовой механикой, после получения основных результатов функана в квантовой механике случился значительный прорыв~\cite{quantum}. Оказалось, что самосопряженные операторы как нельзя лучше подходят для описания изменений в квантовой системе. Более того, понятие <<спектр оператора>> оказалось тесно связанным с физическим спектром.
Применение функционального анализа для оценки погрешностей было предложено только после развития компьютерной техники --- в 40-е годы XX века. Если же говорить про применение функана к стохастическим процессам, то оно началось с квантовой механики. Однако с развитием компьютерной техники появилось множество вероятностных алгоритмов, для анализа которых также были необходимы средства из функционального анализа.

Таким образом, математику и, в частности, функциональный анализ стоит изучатьне не только ради собственного интереса но и для практической пользы.

Наконец, даже если Вы не хотите иметь ничего общего с квантовой физикой, численными методами и вероятностями, Вам все равно необходимо изучать функциональный анализ. Юрий Шполянский, выпускник кафедры КТ 2000 года, доктор физ-мат наук, профессор, сказал, что функциональный анализ --- самый сложный предмет из всей учебной программы, и что хотя он его не применяет на практике, этот предмет, по его мнению, является очень полезным для мозга. Я полностью согласен с этими словами: в IT-индустрии безусловно много направлений, не требующих от программистов знаний в области функционального анализа, однако как можно добиться существенных успехов в IT без хорошо развитого математического мышления? Павел Дуров наверняка не знает функана, но зато его брат Николай знает точно...

Для развития такого мышления мало одного математического предмета в семестр (как у нас это происходит с гуманитарными предметами). Для этого учебная программа должна содержать целый комплекс различных фундаментальных математических дисциплин, включающий в себя не только анализ (математический и функциональный), но и теорию вероятностей, матстатистику, дифференциальные урванения, теорию функций комплексной переменной, теорию чисел и некоторые другие. Только тогда выпускник КТ сможет считать себя человеком с высшим образованием, а не стоять в одном ряду с программистами-самоучками, которые выучили несколько языков программирования с помощью Гугла. Конечно, и самоучки типа Джобса и Гейтса бывают великими, но они вряд ли бы прошли собеседование в своих компаниях. Например, одно из часовых (!) собеседований Ивана Белоногова, когда он поступал на работу в компанию OpenAI~\cite{}, было посвящено теории вероятностей и линейной алгебре.

Надо признать, что в последнее время с развитием технологий растет объем предметов, которые следует преподавать студентам нашего направления. Однако это не значит, что нужно уменьшать объем математических дисциплин в учебной программе, а надо повышать требования к студентам. К нам идут одни из самых талантливых школьников России в надежде получить лучшее образование, и кафедра должна отвечать ожиданиям не только самих студентов, но и их родителей. Не все поступающие могут справляться с нагрузкой, которую предполагает трудное обучение, однако это не является поводом подстраивать программу под них. Существует множество других кафедр, на которых также обучают программистов, но с более простыми учебными планами. Я говорю не только про ИТМО --- в стране есть 450 вузов, в которых готовят <<специалистов>> по IT, которые могут составить большое народное IT-ополчение. А мы же готовим спецназ.

\textbf{Заключение}

Подводя итоги, соберем вместе главные аргументы в пользу необходимости изучения функционального анализа.

\begin{enumerate}
  \item У функана есть множество практических применений~\cite{quantum,kollats,prob}.
  \item Функан является сложным предметом, но это не причина не изучать его.
  \item Функан, как и любая друга математическая дисциплина, может оказаться полезным в самых неожиданных областях.
  \item Функан также вносит неоценимый вклад в развитие математического мышления.
\end{enumerate}

Я очень надеюсь, что данный текст поможет студентам лучше понять, почему им может оказаться нужен.

Автор благодарен Анатолию Абрамовичу Шалыто за инциативу написания этого текста и редактирование его.

\bibliographystyle{plain}
\bibliography{bibliography}

\end{document}

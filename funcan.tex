\documentclass[russian]{article}
\usepackage[T2A,T1]{fontenc}
\usepackage[utf8]{inputenc}
\usepackage[a4paper]{geometry}
\geometry{verbose,tmargin=2cm,bmargin=2cm,lmargin=2cm,rmargin=2cm}
\usepackage{babel}

\title{Про функциональный анализ}

\begin{document}
\maketitle

Анатолий Абрамович недавно попросил меня написать текст на тему того, зачем на кафедре КТ нужен такой предмет как функциональный анализ. Текст должен предназначаться студентам, которые приходят к нему с этим вопросом.

Сначала меня удивил сам факт того, что студенты приходят с таким вопросом к Анатолию Абрамовичу. Причем это удивляет меня не по единственной причине. Во-первых, для меня вопрос "зачем нужен функан на КТ" стоит в одном ряду с вопросом "зачем нужна математика в школе". Он кажется нелепым по той причине, что функциональный анализ является одним из базовых предметов. Это, вполне вероятно, является субъективным мнением, с которым, тем не менее, согласятнся многие выпускники кафедры, но я приведу рассуждения по этому поводу позже.

Второй причиной того, что вопрос студентов меня удивил, было то, что он был задан даже несмотря на то, что в наше время у всех есть доступ к всемирной сети, в том числе к ресурсу Google.com. Потратив некоторое время на пользование этим ресурсом, можно легко найти обсуждения данной темы. Например, по запросу "Why learn functional analysis" легко можно найти рассуждение~\cite{why-funcan}, в котором приведены примеры применения и причины необходимости функционального анализа в учебной программе по прикладной математике. И данное обсуждение является далеко не единственным на данную тему. Даже если же вопрос не столько про функциональный анализ, а более общей, про то, зачем вообще нужна "чистая" математика в учебных программах, то, опять же, по запросу "why we learn pure math" можно найти обсуждения данного вопроса, например,~\cite{why-pure-math}.

Наконец, третья вещь, которая меня удивила, это зачем задавать данный вопрос Анатолию Абрамовичу, когда можно задать его непосредственно преподавателю функционального анализа или на худой конец мне. Мы могли бы посоветовать посмотреть книгу~\cite{kollats}, которая хоть и была издана почти 50 лет назад, но приведенные в книге примеры применения функана актуальны до сих пор.

Несмотря на все это, я все-таки понимаю необходимось написания данного текста. Основной причиной того, что студенты задали такой вопрос Анатолию Абрамовичу, является то, что ответы не агрегированы в одном месте, а разбросаны по разным источникам. Поэтому своей задачей я ставлю именно сбор разобщенной информации в одном месте. Благо дело, наличие различных источников информации позволяет мне практически не генерировать собственные мысли, а просто цитировать более умных людей. Хотя, разумется

На самом деле вопрос необходимости функана куда более глубокий, чем кажется, и уходит корнями в филосовские вопросы о методах познания. Для начала стоит вспомнить, что в новое время появилось два основных направления в философии науки --- эмпиризм и рационализм, которые во многом противопоставляли себя друг другу~\cite{phil}. В основе рацонализма лежит идея возможности логического познания мира, которая берет свое начало еще из "Аналитик" Аристотеля~\cite{aristotel}. Эмпиризм, основателем которого принято считать Бэейкона~\cite{bacon}, напротив, считает возможным только чувственное восприятие мира, и ставит единственным критерием истинности эксперимент. В наше время большинство ученых сходится в том, что ни экспериментальная, ни теоретическая наука не самодостаточны, но дополняют друг друга, и потому обе являются необходимыми для расширения человеческих заний. В качестве примера приведу более подробное рассуждение на эту тему в области эволюционных вычислений, представленной в первой части~\cite{doerr}.

Функциональный анализ является, как это ясно из названия, теоретической наукой, как и многие другие ветви математики, такие как топология, теория чисел, теория игр и другие. Однако несмотря на то, что все эти науки являются неприкладными по своей сути, все они нашли себе применение в прикладных задачах. Топология нашла себе применение в анализе данных (topological data analysis)~\cite{topology}, теория чисел --- в криптографии~\cite{numbers}, теория игр --- в экономике~\cite{games}. Функциональный анализ также нашел себе практическое применение. Самым ярким является его применение в квантовой механике. 

\bibliographystyle{plain}
\bibliography{bibliography}

\end{document}

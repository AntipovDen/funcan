\documentclass[russian]{article}
\usepackage[T2A,T1]{fontenc}
\usepackage[utf8]{inputenc}
\usepackage[a4paper]{geometry}
\geometry{verbose,tmargin=4cm,bmargin=4cm,lmargin=4cm,rmargin=4cm}
\usepackage{babel}

\title{Про функциональный анализ}
\author{Денис Антипов}

\begin{document}
\maketitle

Анатолий Абрамович недавно попросил меня написать текст на тему того, зачем на кафедре КТ нужен такой предмет как функциональный анализ. Текст должен предназначаться студентам, которые приходят к нему с этим вопросом.

Сначала меня удивил сам факт того, что студенты приходят с таким вопросом к Анатолию Абрамовичу. Причем это удивляет меня не по единственной причине. Во-первых, для меня вопрос <<зачем нужен функан на КТ>> стоит в одном ряду с вопросом <<зачем нужна математика в школе>>. Он кажется нелепым по той причине, что функциональный анализ является одним из базовых предметов. Это, вполне вероятно, является субъективным мнением, с которым, тем не менее, согласятся многие выпускники кафедры, но я приведу рассуждения по этому поводу позже.

Второй причиной того, что вопрос студентов меня удивил, было то, что он был задан даже несмотря на то, что в наше время у всех есть доступ к всемирной сети, в том числе к ресурсу Google.com. Потратив некоторое время на пользование этим ресурсом, можно легко найти обсуждения данной темы. Например, по запросу <<Why learn functional analysis>> легко можно найти рассуждение~\cite{why-funcan}, в котором приведены примеры применения и причины необходимости функционального анализа в учебной программе по прикладной математике. И данное обсуждение является далеко не единственным на данную тему. Даже если же вопрос не столько про функциональный анализ, а более общей, про то, зачем вообще нужна чистая математика в учебных программах, то, опять же, по запросу <<why we learn pure math>> можно найти обсуждения данного вопроса, например,~\cite{why-pure-math}.

Наконец, третья вещь, которая меня удивила, это зачем задавать данный вопрос Анатолию Абрамовичу, когда можно задать его непосредственно преподавателю функционального анализа или на худой конец мне. Мы могли бы посоветовать посмотреть книгу~\cite{kollats}, которая хоть и была издана почти 50 лет назад, но приведенные в книге примеры применения функана актуальны до сих пор.

Несмотря на все это, я все-таки понимаю необходимость написания данного текста. Основной причиной того, что студенты задали такой вопрос Анатолию Абрамовичу, является то, что ответы не агрегированы в одном месте, а разбросаны по разным источникам. Поэтому своей задачей я ставлю именно сбор разобщенной информации в одном месте. Благо дело, наличие различных источников информации позволяет мне практически не генерировать собственные мысли, а просто цитировать более умных людей. Хотя ближе к концу я и добавлю некоторые собственные рассуждения.

На самом деле вопрос необходимости функана куда более глубокий, чем кажется, и уходит корнями в философские вопросы о методах познания. Для начала стоит вспомнить, что в новое время появилось два основных направления в философии науки --- эмпиризм и рационализм, которые во многом противопоставляли себя друг другу~\cite{phil}. В основе классического рационализма, главные принципы которого были сформулированы Декартом~\cite{descartes}, лежит идея возможности логического познания мира, которая берет свое начало еще из <<Аналитик>> Аристотеля~\cite{aristotel}. Эмпиризм, основателем которого принято считать Бэкона~\cite{bacon}, напротив, считает возможным только чувственное восприятие мира, и ставит единственным критерием истинности эксперимент. В наше время большинство ученых сходится в том, что ни экспериментальная, ни теоретическая наука не самодостаточны, но дополняют друг друга, и потому обе являются необходимыми для расширения человеческих заний. Более подробное рассуждение на эту тему в области эволюционных вычислений можно найти в первой части~\cite{doerr}. Хотя, разумеется, стоит признать, что до сих пор даже среди ученых встречаются люди, не признающие чисто теоретические или чисто практические работы (первое встречается чаще). В этом я убедился на недавней конференции PPSN 2018, где я потратил минут десять своего доклада на то, чтобы объяснить одному китайцу необходимость теории в области эволюционных вычислений. Однако десяти минут не было достаточно, чтобы изменить мнение убежденного эмпирика, так как, повторюсь, данные вопросы являются глубоко философскими.

Функциональный анализ является, как это ясно из названия, теоретической наукой, как и многие другие ветви математики, такие как топология, теория чисел, теория игр и другие. Однако несмотря на то, что все эти науки являются неприкладными по своей сути, все они нашли себе применение в прикладных задачах. Топология нашла себе применение в анализе данных (TDA --- Topological Data Analysis)~\cite{topology}, теория чисел --- в криптографии~\cite{numbers}, теория игр --- в экономике~\cite{games}. Функциональному анализу также было найдено практическое применение. Самым ярким примером является его применение в квантовой механике~\cite{quantum}. Но, я думаю, большинство современных студентов КТ не считают нужным изучать квантовую механику (и, как это ни грустно, физику в целом), поэтому более близкий пример для КТ-шников --- это применение функана для оценки погрешности вычислений численных методов при решении различных задач, в том числе нелинейных, что описано в книге~\cite{kollats}. В этой же книге содержится множество ссылок на другие работы, посвященные практическому применению функционального анализа. На случай, если примеров применения фукана все еще недостаточно, то он широко применяется в теории вероятностей для анализа стохастических процессов~\cite{prob}. В своей деятельности я пользуюсь функаном именно в этом контексте. Например, в моей последней работе с Бенжамином Доерром~\cite{plateau} знания функционального анализа очень помогали мне осознавать особенности анализируемого стохастического процесса и добиться полученных результатов, по ходу разработав новый метод анализа эволюционных алгоритмов на плато.

Другая причина, почему стоит изучать матанализ, --- историческая. Стоит признать, оглядываясь на историю математики, что большинство известных математических результатов было получено просто потому, что математикам это было интересно, а не потому, что они знали про какое-любо практическое применение своих результатов заранее. Можно снова привести примеры топологии, теории чисел и теории игр, так как сначала появились эти ветви математики, а потом люди нашли им практическое применение. Но наиболее забавным мне кажется пример Джорджа Буля. Буль одним из первых пришел к идее чистой математики, в которой математик оперирует символами, представляющими некоторые объекты, а не самими объектами. Буль утверждал, что математика не должна привязываться к реальным объектам, должна быть абстрактной. Это привело его к разработке матлогики и булевой алгебры в 1847 году~\cite{boole}. И хотя Буль очень хотел, чтобы его алгебра была примером чистой, неприкладной математики, все мы знаем, что после развития вычислительной техники работы Буля стали настолько прикладными, что современный мир просто не мог бы без них существовать.

С функциональным анализом история была примерно такой же. Хоть он и начинал свое развитие примерно в одно время с квантовой механикой, после получения основных результатов функана в квантовой механике случился значительный прорыв~\cite{quantum}. Оказалось, что самосопряженные операторы как нельзя лучше подходят для описания изменений в квантовой системе. Более того, понятие спектра оператора оказалось тесно связанным с физическим спектром.
Что касается применения функционального анализа для оценки погрешностей --- в данной области он также нашел прикладное применение только после развития компьютерной техники, то есть в 40-е годы XX века. Если же говорить про применение функана к стохастическим процессам, то данное применение началось собственно с квантовой механики. Однако с развитием компьютерной техники появилось множество вероятностных алгоритмов, для анализа которых также были необходимы средства из функционального анализа.

Таким образом, мы приходим к выводу, что математику и, в частности, функциональный анализ стоит изучать (и развивать) в том числе из эгоистичных побуждений: ради собственного интереса и удовольствия. А применение может быть найдено само собой позже в самых неожиданных областях. Ведь стоит признать, что лучше владеть ненужным инструментом, чем не иметь нужного инструмента под рукой. Тем более что ненужный инструмент, вероятно, просто ждет своего времени.

Наконец, даже если вы не хотите иметь ничего общего с квантовой физикой, численными методами и вероятностями, вам все равно необходимо изучать функциональный анализ. Юрий Шполянский, выпускник кафедры КТ 2000 года, д.ф-м.н., в разговоре с Анатолием Абрамовичем сказал, что функциональный анализ --- самый сложный предмет из всей учебной программы, и что хотя он его не применяет на практике, данный предмет является очень полезным для мозга. Я полностью согласен с этими словами: в IT индустрии безусловно много направлений, не требующих от программистов знаний в области функционального анализа, однако как можно работать в IT без хорошо развитого математического мышления? Для развития такого мышления мало одного математического предмета в семестре (как у нас это устроено с гуманитарными науками). Для этого учебная программа должна содержать целый комплекс различных фундаментальных математических дисциплин, включающий в себя не только анализ (математический и функциональный), но и теорию вероятностей, матстатистику, дифференциальные урванения, ТФКП, теорию чисел и некоторые другие. Только тогда выпускник КТ сможет считать себя человеком с высшим образованием, а не стоять в одном ряду с программистами-самоучками, которые выучили несколько языков программирования с помощью гугла, потратив при этом значительно меньше шести лет на обучение.

Надо признать, что в последнее время с ростом технологий растет и объем предметов, которые следует преподавать студентам. Однако это не значит, что нужно уменьшать объем математических дисциплин в учебной программе, а значит, что надо повышать требования к студентам. К нам идут самые талантливые школьники России в надежде получить лучшее образование, и кафедра должна отвечать их ожиданиям. Не все поступившие могут справляться с нагрузкой, которую подразумевает обучение, однако это не является поводом подстраивать программу под самых слабых студентов. Существует множество других кафедр (я говорю не только про ИТМО), на которых также обучают программистов, но по более простым программам, которые оставляют студентам гораздо больше свободного времени. Поэтому у любого студента, поступившего на КТ, всегда есть возможность получить высшее образование в другом месте и впоследствии найти хорошую работу на должности программиста. Но для людей, способных преодолеть все трудности учебы на КТ, жизненно необходимо поддерживать сложность учебной программы, чтобы те, кто смог это все осилить и пережить, везде считались программистами на порядок сильнее остальных.

\bibliographystyle{plain}
\bibliography{bibliography}

\end{document}
